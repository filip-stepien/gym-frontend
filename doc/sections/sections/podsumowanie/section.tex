\documentclass[../../spr.tex]{subfiles}

\begin{document}

\section{Podsumowanie}

\subsection{Wnioski z realizacji projektu}

Realizacja aplikacji do zarządzania siłownią pozwoliła zdobyć praktyczne doświadczenie w projektowaniu i implementacji złożonego interfejsu użytkownika z wykorzystaniem nowoczesnych rozwiązań frontendowych. W trakcie prac możliwe było zastosowanie w praktyce wiedzy dotyczącej architektury aplikacji typu SPA (\textit{Single Page Application}), obsługi routingu, stanu aplikacji oraz integracji z zewnętrznymi usługami. Projekt umożliwił także lepsze zrozumienie zasad projektowania komponentów, tworzenia struktury kodu możliwej do wykorzystania w różnych częściach aplikacji oraz skutecznego testowania aplikacji — wcześniej wydawało się, że pisanie testów w aplikacjach \textit{frontendowych} jest mało przydatne lub trudne do zrealizowania. Okazało się, że każdy z tych aspektów jest kluczowy dla prawidłowego działania aplikacji.

\subsection{Ocena osiągniętych rezultatów i refleksje na temat procesu implementacji}

Zrealizowana aplikacja spełnia założone wymagania funkcjonalne i techniczne. Interfejs użytkownika został zaprojektowany w sposób intuicyjny, a podział na role użytkowników (klient, pracownik, trener, menadżer) pozwala na klarowną separację uprawnień i dostępnych funkcji. Największym wyzwaniem okazała się integracja z backendem (spodziewaliśmy się, że pójdzie sprawniej), szczególnie podczas tworzenia i walidacji formularzy.

Pomimo relatywnie niewielkiej liczby funkcjonalności, projekt okazał się trudny i wyjątkowo czasochłonny. Każdy etap — od zaprojektowania interfejsu, przez jego wdrożenie i przetestowanie, aż po integrację z API — wymagał dużo zaangażowania i dokładności. Cały proces jest złożony i wymaga starannego planowania, co pokazuje, jak czasochłonne jest tworzenie nawet pozornie prostych aplikacji \textit{frontendowych}.

\subsection{Propozycje usprawnień lub dalszego rozwoju aplikacji}

Aplikacja ma duży potencjał dalszego rozwoju. Do możliwych usprawnień należą:

\begin{itemize}
  \item Wprowadzenie systemu powiadomień dla użytkowników (np. przypomnienia o wygasającym karnecie).
  \item Umożliwienie komunikacji pomiędzy trenerem a klientem (chat, wiadomości).
  \item Rozbudowa sekcji analitycznej dla menadżera, np. o raporty frekwencji czy obłożenia sal.
  \item Obsługa różnych rodzajów karnetów.
  \item Dodanie możliwości tworzenia i zarządzania kontami użytkowników bezpośrednio w aplikacji, zamiast manipulowania nimi w systemie \textit{Keycloak}.
  \item Implementacja systemu ocen i opinii na temat trenerów i zajęć.
\end{itemize}

\end{document}
