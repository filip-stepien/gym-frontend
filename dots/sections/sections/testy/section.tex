\documentclass[../../spr.tex]{subfiles}

\begin{document}

\section{Testy}
\subsection{Opis metod testowania (np. testy manualne i automatyczne)}
W projekcie zastosowano kombinację testów manualnych i automatycznych w celu zapewnienia jakości oprogramowania. Testy automatyczne skupiają się na komponentach interfejsu użytkownika, sprawdzając ich zachowanie w różnych scenariuszach. Testy manualne wykorzystano do weryfikacji ogólnej funkcjonalności systemu oraz integracji między modułami.

\subsection{Wyniki testów, napotkane błędy oraz zastosowane rozwiązania}
Przeprowadzone testy obejmowały następujące komponenty:

\begin{itemize}
    \item \textbf{ClientDetailsCard}:
    \begin{itemize}
        \item Weryfikacja renderowania pól z początkowymi wartościami - \textbf{SUKCES}
        \item Sprawdzenie wywołania funkcji \texttt{outfall} z poprawnymi wartościami po wysłaniu formularza - \textbf{SUKCES}
        \item Test braku wywołania \texttt{outfall} gdy wymagane pola są puste - \textbf{SUKCES}
    \end{itemize}
    
    \item \textbf{CurrentMaintenanceTasksCard}:
    \begin{itemize}
        \item Sprawdzenie renderowania wiersza tabeli z informacjami o hali - \textbf{SUKCES}
    \end{itemize}
    
    \item \textbf{EmployeeDetailsCard}:
    \begin{itemize}
        \item Weryfikacja poprawnego renderowania szczegółów pracownika - \textbf{SUKCES}
    \end{itemize}
    
    \item \textbf{EmployeeTableCard}:
    \begin{itemize}
        \item Test renderowania listy pracowników - \textbf{SUKCES}
        \item Sprawdzenie warunkowego renderowania przycisku "+ New" - \textbf{SUKCES}
    \end{itemize}
\end{itemize}

Podczas testów nie zidentyfikowano poważnych błędów. Wszystkie główne funkcjonalności działały zgodnie z oczekiwaniami. Jedyną napotkaną trudnością było warunkowe renderowanie przycisków, które zostało poprawione poprzez dodanie odpowiednich walidacji właściwości komponentów.

\end{document}