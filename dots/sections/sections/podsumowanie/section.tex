\documentclass[../../spr.tex]{subfiles}

\begin{document}

\section{Podsumowanie}

\subsection{Wnioski z realizacji projektu}

Realizacja aplikacji do zarządzania siłownią pozwoliła nam zdobyć praktyczne doświadczenie
w pracy zespołowej nad rzeczywistym projektem informatycznym.
Dzięki podziałowi ról i zadań byliśmy w stanie efektywnie wdrożyć rozbudowaną
aplikację opartą na architekturze klient-serwer.
Projekt umożliwił nam również wykorzystanie nowoczesnych narzędzi
frontendowych oraz automatyzujących tworzenie.

\subsection{Ocena osiągniętych rezultatów i refleksje na temat procesu implementacji}

Zrealizowana aplikacja spełnia założone wymagania funkcjonalne i techniczne.
Interfejs użytkownika został zaprojektowany w sposób intuicyjny,
a podział na role użytkowników (klient, pracownik, trener, menadżer)
pozwala na klarowną separację uprawnień i dostępnych funkcji.
Największym wyzwaniem okazała się integracja z backendem (spodziewaliśmy się, ze pójdzie sprawniej),
szczególnie przy tworzeniu i walidacji formularzy.
Proces ten pozwolił nam lepiej zrozumieć mechanizmy działania
aplikacji typu SPA (Single Page Application).

\subsection{Propozycje usprawnień lub dalszego rozwoju aplikacji}

Aplikacja ma duży potencjał dalszego rozwoju. Do możliwych usprawnień należą:

\begin{itemize}
  \item Wprowadzenie systemu powiadomień dla użytkowników (np. przypomnienia o wygasającym karnecie).
  \item Umożliwienie komunikacji pomiędzy trenerem a klientem (chat, wiadomości).
  \item Rozbudowa sekcji analitycznej dla menadżera, np. o raporty frekwencji czy obłożenia sal.
\end{itemize}
\end{document}
